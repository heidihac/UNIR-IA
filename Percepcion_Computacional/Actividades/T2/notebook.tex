
% Default to the notebook output style

    


% Inherit from the specified cell style.




    
\documentclass[11pt]{article}

    
    
    \usepackage[T1]{fontenc}
    % Nicer default font (+ math font) than Computer Modern for most use cases
    \usepackage{mathpazo}

    % Basic figure setup, for now with no caption control since it's done
    % automatically by Pandoc (which extracts ![](path) syntax from Markdown).
    \usepackage{graphicx}
    % We will generate all images so they have a width \maxwidth. This means
    % that they will get their normal width if they fit onto the page, but
    % are scaled down if they would overflow the margins.
    \makeatletter
    \def\maxwidth{\ifdim\Gin@nat@width>\linewidth\linewidth
    \else\Gin@nat@width\fi}
    \makeatother
    \let\Oldincludegraphics\includegraphics
    % Set max figure width to be 80% of text width, for now hardcoded.
    \renewcommand{\includegraphics}[1]{\Oldincludegraphics[width=.8\maxwidth]{#1}}
    % Ensure that by default, figures have no caption (until we provide a
    % proper Figure object with a Caption API and a way to capture that
    % in the conversion process - todo).
    \usepackage{caption}
    \DeclareCaptionLabelFormat{nolabel}{}
    \captionsetup{labelformat=nolabel}

    \usepackage{adjustbox} % Used to constrain images to a maximum size 
    \usepackage{xcolor} % Allow colors to be defined
    \usepackage{enumerate} % Needed for markdown enumerations to work
    \usepackage{geometry} % Used to adjust the document margins
    \usepackage{amsmath} % Equations
    \usepackage{amssymb} % Equations
    \usepackage{textcomp} % defines textquotesingle
    % Hack from http://tex.stackexchange.com/a/47451/13684:
    \AtBeginDocument{%
        \def\PYZsq{\textquotesingle}% Upright quotes in Pygmentized code
    }
    \usepackage{upquote} % Upright quotes for verbatim code
    \usepackage{eurosym} % defines \euro
    \usepackage[mathletters]{ucs} % Extended unicode (utf-8) support
    \usepackage[utf8x]{inputenc} % Allow utf-8 characters in the tex document
    \usepackage{fancyvrb} % verbatim replacement that allows latex
    \usepackage{grffile} % extends the file name processing of package graphics 
                         % to support a larger range 
    % The hyperref package gives us a pdf with properly built
    % internal navigation ('pdf bookmarks' for the table of contents,
    % internal cross-reference links, web links for URLs, etc.)
    \usepackage{hyperref}
    \usepackage{longtable} % longtable support required by pandoc >1.10
    \usepackage{booktabs}  % table support for pandoc > 1.12.2
    \usepackage[inline]{enumitem} % IRkernel/repr support (it uses the enumerate* environment)
    \usepackage[normalem]{ulem} % ulem is needed to support strikethroughs (\sout)
                                % normalem makes italics be italics, not underlines
    

    
    
    % Colors for the hyperref package
    \definecolor{urlcolor}{rgb}{0,.145,.698}
    \definecolor{linkcolor}{rgb}{.71,0.21,0.01}
    \definecolor{citecolor}{rgb}{.12,.54,.11}

    % ANSI colors
    \definecolor{ansi-black}{HTML}{3E424D}
    \definecolor{ansi-black-intense}{HTML}{282C36}
    \definecolor{ansi-red}{HTML}{E75C58}
    \definecolor{ansi-red-intense}{HTML}{B22B31}
    \definecolor{ansi-green}{HTML}{00A250}
    \definecolor{ansi-green-intense}{HTML}{007427}
    \definecolor{ansi-yellow}{HTML}{DDB62B}
    \definecolor{ansi-yellow-intense}{HTML}{B27D12}
    \definecolor{ansi-blue}{HTML}{208FFB}
    \definecolor{ansi-blue-intense}{HTML}{0065CA}
    \definecolor{ansi-magenta}{HTML}{D160C4}
    \definecolor{ansi-magenta-intense}{HTML}{A03196}
    \definecolor{ansi-cyan}{HTML}{60C6C8}
    \definecolor{ansi-cyan-intense}{HTML}{258F8F}
    \definecolor{ansi-white}{HTML}{C5C1B4}
    \definecolor{ansi-white-intense}{HTML}{A1A6B2}

    % commands and environments needed by pandoc snippets
    % extracted from the output of `pandoc -s`
    \providecommand{\tightlist}{%
      \setlength{\itemsep}{0pt}\setlength{\parskip}{0pt}}
    \DefineVerbatimEnvironment{Highlighting}{Verbatim}{commandchars=\\\{\}}
    % Add ',fontsize=\small' for more characters per line
    \newenvironment{Shaded}{}{}
    \newcommand{\KeywordTok}[1]{\textcolor[rgb]{0.00,0.44,0.13}{\textbf{{#1}}}}
    \newcommand{\DataTypeTok}[1]{\textcolor[rgb]{0.56,0.13,0.00}{{#1}}}
    \newcommand{\DecValTok}[1]{\textcolor[rgb]{0.25,0.63,0.44}{{#1}}}
    \newcommand{\BaseNTok}[1]{\textcolor[rgb]{0.25,0.63,0.44}{{#1}}}
    \newcommand{\FloatTok}[1]{\textcolor[rgb]{0.25,0.63,0.44}{{#1}}}
    \newcommand{\CharTok}[1]{\textcolor[rgb]{0.25,0.44,0.63}{{#1}}}
    \newcommand{\StringTok}[1]{\textcolor[rgb]{0.25,0.44,0.63}{{#1}}}
    \newcommand{\CommentTok}[1]{\textcolor[rgb]{0.38,0.63,0.69}{\textit{{#1}}}}
    \newcommand{\OtherTok}[1]{\textcolor[rgb]{0.00,0.44,0.13}{{#1}}}
    \newcommand{\AlertTok}[1]{\textcolor[rgb]{1.00,0.00,0.00}{\textbf{{#1}}}}
    \newcommand{\FunctionTok}[1]{\textcolor[rgb]{0.02,0.16,0.49}{{#1}}}
    \newcommand{\RegionMarkerTok}[1]{{#1}}
    \newcommand{\ErrorTok}[1]{\textcolor[rgb]{1.00,0.00,0.00}{\textbf{{#1}}}}
    \newcommand{\NormalTok}[1]{{#1}}
    
    % Additional commands for more recent versions of Pandoc
    \newcommand{\ConstantTok}[1]{\textcolor[rgb]{0.53,0.00,0.00}{{#1}}}
    \newcommand{\SpecialCharTok}[1]{\textcolor[rgb]{0.25,0.44,0.63}{{#1}}}
    \newcommand{\VerbatimStringTok}[1]{\textcolor[rgb]{0.25,0.44,0.63}{{#1}}}
    \newcommand{\SpecialStringTok}[1]{\textcolor[rgb]{0.73,0.40,0.53}{{#1}}}
    \newcommand{\ImportTok}[1]{{#1}}
    \newcommand{\DocumentationTok}[1]{\textcolor[rgb]{0.73,0.13,0.13}{\textit{{#1}}}}
    \newcommand{\AnnotationTok}[1]{\textcolor[rgb]{0.38,0.63,0.69}{\textbf{\textit{{#1}}}}}
    \newcommand{\CommentVarTok}[1]{\textcolor[rgb]{0.38,0.63,0.69}{\textbf{\textit{{#1}}}}}
    \newcommand{\VariableTok}[1]{\textcolor[rgb]{0.10,0.09,0.49}{{#1}}}
    \newcommand{\ControlFlowTok}[1]{\textcolor[rgb]{0.00,0.44,0.13}{\textbf{{#1}}}}
    \newcommand{\OperatorTok}[1]{\textcolor[rgb]{0.40,0.40,0.40}{{#1}}}
    \newcommand{\BuiltInTok}[1]{{#1}}
    \newcommand{\ExtensionTok}[1]{{#1}}
    \newcommand{\PreprocessorTok}[1]{\textcolor[rgb]{0.74,0.48,0.00}{{#1}}}
    \newcommand{\AttributeTok}[1]{\textcolor[rgb]{0.49,0.56,0.16}{{#1}}}
    \newcommand{\InformationTok}[1]{\textcolor[rgb]{0.38,0.63,0.69}{\textbf{\textit{{#1}}}}}
    \newcommand{\WarningTok}[1]{\textcolor[rgb]{0.38,0.63,0.69}{\textbf{\textit{{#1}}}}}
    
    
    % Define a nice break command that doesn't care if a line doesn't already
    % exist.
    \def\br{\hspace*{\fill} \\* }
    % Math Jax compatability definitions
    \def\gt{>}
    \def\lt{<}
    % Document parameters
    \title{T2-Manuel\_Pasieka-Detecci?n de texturas}
    
    
    

    % Pygments definitions
    
\makeatletter
\def\PY@reset{\let\PY@it=\relax \let\PY@bf=\relax%
    \let\PY@ul=\relax \let\PY@tc=\relax%
    \let\PY@bc=\relax \let\PY@ff=\relax}
\def\PY@tok#1{\csname PY@tok@#1\endcsname}
\def\PY@toks#1+{\ifx\relax#1\empty\else%
    \PY@tok{#1}\expandafter\PY@toks\fi}
\def\PY@do#1{\PY@bc{\PY@tc{\PY@ul{%
    \PY@it{\PY@bf{\PY@ff{#1}}}}}}}
\def\PY#1#2{\PY@reset\PY@toks#1+\relax+\PY@do{#2}}

\expandafter\def\csname PY@tok@w\endcsname{\def\PY@tc##1{\textcolor[rgb]{0.73,0.73,0.73}{##1}}}
\expandafter\def\csname PY@tok@c\endcsname{\let\PY@it=\textit\def\PY@tc##1{\textcolor[rgb]{0.25,0.50,0.50}{##1}}}
\expandafter\def\csname PY@tok@cp\endcsname{\def\PY@tc##1{\textcolor[rgb]{0.74,0.48,0.00}{##1}}}
\expandafter\def\csname PY@tok@k\endcsname{\let\PY@bf=\textbf\def\PY@tc##1{\textcolor[rgb]{0.00,0.50,0.00}{##1}}}
\expandafter\def\csname PY@tok@kp\endcsname{\def\PY@tc##1{\textcolor[rgb]{0.00,0.50,0.00}{##1}}}
\expandafter\def\csname PY@tok@kt\endcsname{\def\PY@tc##1{\textcolor[rgb]{0.69,0.00,0.25}{##1}}}
\expandafter\def\csname PY@tok@o\endcsname{\def\PY@tc##1{\textcolor[rgb]{0.40,0.40,0.40}{##1}}}
\expandafter\def\csname PY@tok@ow\endcsname{\let\PY@bf=\textbf\def\PY@tc##1{\textcolor[rgb]{0.67,0.13,1.00}{##1}}}
\expandafter\def\csname PY@tok@nb\endcsname{\def\PY@tc##1{\textcolor[rgb]{0.00,0.50,0.00}{##1}}}
\expandafter\def\csname PY@tok@nf\endcsname{\def\PY@tc##1{\textcolor[rgb]{0.00,0.00,1.00}{##1}}}
\expandafter\def\csname PY@tok@nc\endcsname{\let\PY@bf=\textbf\def\PY@tc##1{\textcolor[rgb]{0.00,0.00,1.00}{##1}}}
\expandafter\def\csname PY@tok@nn\endcsname{\let\PY@bf=\textbf\def\PY@tc##1{\textcolor[rgb]{0.00,0.00,1.00}{##1}}}
\expandafter\def\csname PY@tok@ne\endcsname{\let\PY@bf=\textbf\def\PY@tc##1{\textcolor[rgb]{0.82,0.25,0.23}{##1}}}
\expandafter\def\csname PY@tok@nv\endcsname{\def\PY@tc##1{\textcolor[rgb]{0.10,0.09,0.49}{##1}}}
\expandafter\def\csname PY@tok@no\endcsname{\def\PY@tc##1{\textcolor[rgb]{0.53,0.00,0.00}{##1}}}
\expandafter\def\csname PY@tok@nl\endcsname{\def\PY@tc##1{\textcolor[rgb]{0.63,0.63,0.00}{##1}}}
\expandafter\def\csname PY@tok@ni\endcsname{\let\PY@bf=\textbf\def\PY@tc##1{\textcolor[rgb]{0.60,0.60,0.60}{##1}}}
\expandafter\def\csname PY@tok@na\endcsname{\def\PY@tc##1{\textcolor[rgb]{0.49,0.56,0.16}{##1}}}
\expandafter\def\csname PY@tok@nt\endcsname{\let\PY@bf=\textbf\def\PY@tc##1{\textcolor[rgb]{0.00,0.50,0.00}{##1}}}
\expandafter\def\csname PY@tok@nd\endcsname{\def\PY@tc##1{\textcolor[rgb]{0.67,0.13,1.00}{##1}}}
\expandafter\def\csname PY@tok@s\endcsname{\def\PY@tc##1{\textcolor[rgb]{0.73,0.13,0.13}{##1}}}
\expandafter\def\csname PY@tok@sd\endcsname{\let\PY@it=\textit\def\PY@tc##1{\textcolor[rgb]{0.73,0.13,0.13}{##1}}}
\expandafter\def\csname PY@tok@si\endcsname{\let\PY@bf=\textbf\def\PY@tc##1{\textcolor[rgb]{0.73,0.40,0.53}{##1}}}
\expandafter\def\csname PY@tok@se\endcsname{\let\PY@bf=\textbf\def\PY@tc##1{\textcolor[rgb]{0.73,0.40,0.13}{##1}}}
\expandafter\def\csname PY@tok@sr\endcsname{\def\PY@tc##1{\textcolor[rgb]{0.73,0.40,0.53}{##1}}}
\expandafter\def\csname PY@tok@ss\endcsname{\def\PY@tc##1{\textcolor[rgb]{0.10,0.09,0.49}{##1}}}
\expandafter\def\csname PY@tok@sx\endcsname{\def\PY@tc##1{\textcolor[rgb]{0.00,0.50,0.00}{##1}}}
\expandafter\def\csname PY@tok@m\endcsname{\def\PY@tc##1{\textcolor[rgb]{0.40,0.40,0.40}{##1}}}
\expandafter\def\csname PY@tok@gh\endcsname{\let\PY@bf=\textbf\def\PY@tc##1{\textcolor[rgb]{0.00,0.00,0.50}{##1}}}
\expandafter\def\csname PY@tok@gu\endcsname{\let\PY@bf=\textbf\def\PY@tc##1{\textcolor[rgb]{0.50,0.00,0.50}{##1}}}
\expandafter\def\csname PY@tok@gd\endcsname{\def\PY@tc##1{\textcolor[rgb]{0.63,0.00,0.00}{##1}}}
\expandafter\def\csname PY@tok@gi\endcsname{\def\PY@tc##1{\textcolor[rgb]{0.00,0.63,0.00}{##1}}}
\expandafter\def\csname PY@tok@gr\endcsname{\def\PY@tc##1{\textcolor[rgb]{1.00,0.00,0.00}{##1}}}
\expandafter\def\csname PY@tok@ge\endcsname{\let\PY@it=\textit}
\expandafter\def\csname PY@tok@gs\endcsname{\let\PY@bf=\textbf}
\expandafter\def\csname PY@tok@gp\endcsname{\let\PY@bf=\textbf\def\PY@tc##1{\textcolor[rgb]{0.00,0.00,0.50}{##1}}}
\expandafter\def\csname PY@tok@go\endcsname{\def\PY@tc##1{\textcolor[rgb]{0.53,0.53,0.53}{##1}}}
\expandafter\def\csname PY@tok@gt\endcsname{\def\PY@tc##1{\textcolor[rgb]{0.00,0.27,0.87}{##1}}}
\expandafter\def\csname PY@tok@err\endcsname{\def\PY@bc##1{\setlength{\fboxsep}{0pt}\fcolorbox[rgb]{1.00,0.00,0.00}{1,1,1}{\strut ##1}}}
\expandafter\def\csname PY@tok@kc\endcsname{\let\PY@bf=\textbf\def\PY@tc##1{\textcolor[rgb]{0.00,0.50,0.00}{##1}}}
\expandafter\def\csname PY@tok@kd\endcsname{\let\PY@bf=\textbf\def\PY@tc##1{\textcolor[rgb]{0.00,0.50,0.00}{##1}}}
\expandafter\def\csname PY@tok@kn\endcsname{\let\PY@bf=\textbf\def\PY@tc##1{\textcolor[rgb]{0.00,0.50,0.00}{##1}}}
\expandafter\def\csname PY@tok@kr\endcsname{\let\PY@bf=\textbf\def\PY@tc##1{\textcolor[rgb]{0.00,0.50,0.00}{##1}}}
\expandafter\def\csname PY@tok@bp\endcsname{\def\PY@tc##1{\textcolor[rgb]{0.00,0.50,0.00}{##1}}}
\expandafter\def\csname PY@tok@fm\endcsname{\def\PY@tc##1{\textcolor[rgb]{0.00,0.00,1.00}{##1}}}
\expandafter\def\csname PY@tok@vc\endcsname{\def\PY@tc##1{\textcolor[rgb]{0.10,0.09,0.49}{##1}}}
\expandafter\def\csname PY@tok@vg\endcsname{\def\PY@tc##1{\textcolor[rgb]{0.10,0.09,0.49}{##1}}}
\expandafter\def\csname PY@tok@vi\endcsname{\def\PY@tc##1{\textcolor[rgb]{0.10,0.09,0.49}{##1}}}
\expandafter\def\csname PY@tok@vm\endcsname{\def\PY@tc##1{\textcolor[rgb]{0.10,0.09,0.49}{##1}}}
\expandafter\def\csname PY@tok@sa\endcsname{\def\PY@tc##1{\textcolor[rgb]{0.73,0.13,0.13}{##1}}}
\expandafter\def\csname PY@tok@sb\endcsname{\def\PY@tc##1{\textcolor[rgb]{0.73,0.13,0.13}{##1}}}
\expandafter\def\csname PY@tok@sc\endcsname{\def\PY@tc##1{\textcolor[rgb]{0.73,0.13,0.13}{##1}}}
\expandafter\def\csname PY@tok@dl\endcsname{\def\PY@tc##1{\textcolor[rgb]{0.73,0.13,0.13}{##1}}}
\expandafter\def\csname PY@tok@s2\endcsname{\def\PY@tc##1{\textcolor[rgb]{0.73,0.13,0.13}{##1}}}
\expandafter\def\csname PY@tok@sh\endcsname{\def\PY@tc##1{\textcolor[rgb]{0.73,0.13,0.13}{##1}}}
\expandafter\def\csname PY@tok@s1\endcsname{\def\PY@tc##1{\textcolor[rgb]{0.73,0.13,0.13}{##1}}}
\expandafter\def\csname PY@tok@mb\endcsname{\def\PY@tc##1{\textcolor[rgb]{0.40,0.40,0.40}{##1}}}
\expandafter\def\csname PY@tok@mf\endcsname{\def\PY@tc##1{\textcolor[rgb]{0.40,0.40,0.40}{##1}}}
\expandafter\def\csname PY@tok@mh\endcsname{\def\PY@tc##1{\textcolor[rgb]{0.40,0.40,0.40}{##1}}}
\expandafter\def\csname PY@tok@mi\endcsname{\def\PY@tc##1{\textcolor[rgb]{0.40,0.40,0.40}{##1}}}
\expandafter\def\csname PY@tok@il\endcsname{\def\PY@tc##1{\textcolor[rgb]{0.40,0.40,0.40}{##1}}}
\expandafter\def\csname PY@tok@mo\endcsname{\def\PY@tc##1{\textcolor[rgb]{0.40,0.40,0.40}{##1}}}
\expandafter\def\csname PY@tok@ch\endcsname{\let\PY@it=\textit\def\PY@tc##1{\textcolor[rgb]{0.25,0.50,0.50}{##1}}}
\expandafter\def\csname PY@tok@cm\endcsname{\let\PY@it=\textit\def\PY@tc##1{\textcolor[rgb]{0.25,0.50,0.50}{##1}}}
\expandafter\def\csname PY@tok@cpf\endcsname{\let\PY@it=\textit\def\PY@tc##1{\textcolor[rgb]{0.25,0.50,0.50}{##1}}}
\expandafter\def\csname PY@tok@c1\endcsname{\let\PY@it=\textit\def\PY@tc##1{\textcolor[rgb]{0.25,0.50,0.50}{##1}}}
\expandafter\def\csname PY@tok@cs\endcsname{\let\PY@it=\textit\def\PY@tc##1{\textcolor[rgb]{0.25,0.50,0.50}{##1}}}

\def\PYZbs{\char`\\}
\def\PYZus{\char`\_}
\def\PYZob{\char`\{}
\def\PYZcb{\char`\}}
\def\PYZca{\char`\^}
\def\PYZam{\char`\&}
\def\PYZlt{\char`\<}
\def\PYZgt{\char`\>}
\def\PYZsh{\char`\#}
\def\PYZpc{\char`\%}
\def\PYZdl{\char`\$}
\def\PYZhy{\char`\-}
\def\PYZsq{\char`\'}
\def\PYZdq{\char`\"}
\def\PYZti{\char`\~}
% for compatibility with earlier versions
\def\PYZat{@}
\def\PYZlb{[}
\def\PYZrb{]}
\makeatother


    % Exact colors from NB
    \definecolor{incolor}{rgb}{0.0, 0.0, 0.5}
    \definecolor{outcolor}{rgb}{0.545, 0.0, 0.0}



    
    % Prevent overflowing lines due to hard-to-break entities
    \sloppy 
    % Setup hyperref package
    \hypersetup{
      breaklinks=true,  % so long urls are correctly broken across lines
      colorlinks=true,
      urlcolor=urlcolor,
      linkcolor=linkcolor,
      citecolor=citecolor,
      }
    % Slightly bigger margins than the latex defaults
    
    \geometry{verbose,tmargin=1in,bmargin=1in,lmargin=1in,rmargin=1in}
    
    

    \begin{document}
    
    
    \maketitle
    
    

    
    \hypertarget{ejecicio}{%
\section{Ejecicio}\label{ejecicio}}

Deberás hacer uso de la imagen que se facilita. En ella se aprecian
diferentes tipos de texturas (circulares y trapezoidales).
Deberás~distinguir de forma automática y con algoritmos~basados en
entropía y en procesamiento de imágenes el mayor número posible de
texturas. Para ello, se aconseja que sigan las siguientes tareas:

\begin{verbatim}
• Implementar una función de entropía para imágenes, tal y como se ha descrito en la teoría.
• Calcular esa entropía para conjuntos de píxeles de tamaño variable, comenzando por los de 3x3 píxeles.
• Se repetirá el ejercicio para tamaños de 5x5 y 7x7.
• Por último, se representará el resultado obtenido para cada tipo de conjunto y se ha de indicar una evaluación sobre si la entropía ayuda o facilita la extracción de características.
\end{verbatim}

    \begin{Verbatim}[commandchars=\\\{\}]
{\color{incolor}In [{\color{incolor}1}]:} \PY{c+c1}{\PYZsh{}Import matplotlib for easy plotting}
        \PY{k+kn}{import} \PY{n+nn}{matplotlib}
        \PY{k+kn}{from} \PY{n+nn}{matplotlib} \PY{k}{import} \PY{n}{pyplot} \PY{k}{as} \PY{n}{plt}
        
        \PY{c+c1}{\PYZsh{} make matplotlib display figures as part of the notebook}
        \PY{o}{\PYZpc{}}\PY{k}{matplotlib} inline
        
        \PY{c+c1}{\PYZsh{} Load libraries}
        \PY{k+kn}{import} \PY{n+nn}{numpy} \PY{k}{as} \PY{n+nn}{np}
        \PY{k+kn}{import} \PY{n+nn}{skimage}
        
        \PY{c+c1}{\PYZsh{} Load helper modules from skimage}
        \PY{k+kn}{from} \PY{n+nn}{skimage} \PY{k}{import} \PY{n}{data}\PY{p}{,} \PY{n}{io} 
        
        \PY{c+c1}{\PYZsh{} File operaitons}
        \PY{k+kn}{import} \PY{n+nn}{os}
\end{Verbatim}


    \begin{Verbatim}[commandchars=\\\{\}]
{\color{incolor}In [{\color{incolor}2}]:} \PY{c+c1}{\PYZsh{} Load an image from disk}
        \PY{n}{image} \PY{o}{=} \PY{n}{io}\PY{o}{.}\PY{n}{imread}\PY{p}{(}\PY{n}{os}\PY{o}{.}\PY{n}{path}\PY{o}{.}\PY{n}{join}\PY{p}{(}\PY{l+s+s1}{\PYZsq{}}\PY{l+s+s1}{images}\PY{l+s+s1}{\PYZsq{}}\PY{p}{,} \PY{l+s+s1}{\PYZsq{}}\PY{l+s+s1}{tema5\PYZus{}actividad.png}\PY{l+s+s1}{\PYZsq{}}\PY{p}{)}\PY{p}{)}
        
        \PY{c+c1}{\PYZsh{} Display the image}
        \PY{n}{io}\PY{o}{.}\PY{n}{imshow}\PY{p}{(}\PY{n}{image}\PY{p}{)}
\end{Verbatim}


\begin{Verbatim}[commandchars=\\\{\}]
{\color{outcolor}Out[{\color{outcolor}2}]:} <matplotlib.image.AxesImage at 0xa22cc8f28>
\end{Verbatim}
            
    \begin{center}
    \adjustimage{max size={0.9\linewidth}{0.9\paperheight}}{output_2_1.png}
    \end{center}
    { \hspace*{\fill} \\}
    
    \begin{Verbatim}[commandchars=\\\{\}]
{\color{incolor}In [{\color{incolor}3}]:} \PY{c+c1}{\PYZsh{} Convert image to grayscale to make things easier}
        \PY{n}{gimage} \PY{o}{=} \PY{n}{skimage}\PY{o}{.}\PY{n}{color}\PY{o}{.}\PY{n}{rgb2gray}\PY{p}{(}\PY{n}{image}\PY{p}{)}
        \PY{n}{io}\PY{o}{.}\PY{n}{imshow}\PY{p}{(}\PY{n}{gimage}\PY{p}{)}
\end{Verbatim}


\begin{Verbatim}[commandchars=\\\{\}]
{\color{outcolor}Out[{\color{outcolor}3}]:} <matplotlib.image.AxesImage at 0xa22df7630>
\end{Verbatim}
            
    \begin{center}
    \adjustimage{max size={0.9\linewidth}{0.9\paperheight}}{output_3_1.png}
    \end{center}
    { \hspace*{\fill} \\}
    
    \begin{Verbatim}[commandchars=\\\{\}]
{\color{incolor}In [{\color{incolor}4}]:} \PY{c+c1}{\PYZsh{} Have a look at the histogram}
        \PY{n}{plt}\PY{o}{.}\PY{n}{hist}\PY{p}{(}\PY{n}{gimage}\PY{o}{.}\PY{n}{ravel}\PY{p}{(}\PY{p}{)}\PY{p}{,} \PY{n}{bins}\PY{o}{=}\PY{l+m+mi}{100}\PY{p}{)}
        \PY{n}{plt}\PY{o}{.}\PY{n}{title}\PY{p}{(}\PY{l+s+s1}{\PYZsq{}}\PY{l+s+s1}{Image histogram}\PY{l+s+s1}{\PYZsq{}}\PY{p}{)}
        \PY{n}{plt}\PY{o}{.}\PY{n}{show}\PY{p}{(}\PY{p}{)}
\end{Verbatim}


    \begin{center}
    \adjustimage{max size={0.9\linewidth}{0.9\paperheight}}{output_4_0.png}
    \end{center}
    { \hspace*{\fill} \\}
    
    \begin{Verbatim}[commandchars=\\\{\}]
{\color{incolor}In [{\color{incolor}10}]:} \PY{c+c1}{\PYZsh{} Perform a covolution of an image with given function and kernel sizes}
         \PY{k}{def} \PY{n+nf}{convolute\PYZus{}image}\PY{p}{(}\PY{n}{image}\PY{p}{,} \PY{n}{function}\PY{p}{,} \PY{n}{kernels}\PY{o}{=}\PY{p}{[}\PY{p}{(}\PY{l+m+mi}{3}\PY{p}{,} \PY{l+m+mi}{3}\PY{p}{)}\PY{p}{,} \PY{p}{(}\PY{l+m+mi}{5}\PY{p}{,} \PY{l+m+mi}{5}\PY{p}{)}\PY{p}{,} \PY{p}{(}\PY{l+m+mi}{7}\PY{p}{,} \PY{l+m+mi}{7}\PY{p}{)}\PY{p}{]}\PY{p}{)}\PY{p}{:}
             \PY{n}{result} \PY{o}{=} \PY{p}{[}\PY{p}{]}
             \PY{k}{for} \PY{n}{k} \PY{o+ow}{in} \PY{n}{kernels}\PY{p}{:}
                 \PY{n}{entropy\PYZus{}image} \PY{o}{=} \PY{n}{convolution}\PY{p}{(}\PY{n}{image}\PY{p}{,} \PY{n}{function}\PY{p}{,} \PY{n}{k}\PY{p}{)}
                 \PY{n}{result}\PY{o}{.}\PY{n}{append}\PY{p}{(}\PY{n}{normalize}\PY{p}{(}\PY{n}{entropy\PYZus{}image}\PY{p}{)}\PY{p}{)}
             \PY{k}{return} \PY{n}{result}
         
         \PY{c+c1}{\PYZsh{} apply func to a sliding window over the given image}
         \PY{k}{def} \PY{n+nf}{convolution}\PY{p}{(}\PY{n}{img}\PY{p}{,} \PY{n}{func}\PY{p}{,} \PY{n}{sw}\PY{p}{,} \PY{n}{keep\PYZus{}size}\PY{o}{=}\PY{k+kc}{False}\PY{p}{)}\PY{p}{:}
             \PY{n}{bw}\PY{p}{,} \PY{n}{bh} \PY{o}{=} \PY{n}{sw}
             \PY{n}{xb} \PY{o}{=} \PY{n+nb}{int}\PY{p}{(}\PY{p}{(}\PY{n}{bw}\PY{o}{\PYZhy{}}\PY{l+m+mi}{1}\PY{p}{)}\PY{o}{/}\PY{l+m+mi}{2}\PY{p}{)}
             \PY{n}{yb} \PY{o}{=} \PY{n+nb}{int}\PY{p}{(}\PY{p}{(}\PY{n}{bh}\PY{o}{\PYZhy{}}\PY{l+m+mi}{1}\PY{p}{)}\PY{o}{/}\PY{l+m+mi}{2}\PY{p}{)}
             \PY{n}{result} \PY{o}{=} \PY{n}{np}\PY{o}{.}\PY{n}{zeros}\PY{p}{(}\PY{n}{shape}\PY{o}{=}\PY{n}{img}\PY{o}{.}\PY{n}{shape}\PY{p}{)}
             
             \PY{k}{for} \PY{n}{yi} \PY{o+ow}{in} \PY{n+nb}{range}\PY{p}{(}\PY{n}{yb}\PY{p}{,} \PY{n}{img}\PY{o}{.}\PY{n}{shape}\PY{p}{[}\PY{l+m+mi}{0}\PY{p}{]}\PY{o}{\PYZhy{}}\PY{n}{yb}\PY{p}{)}\PY{p}{:}
                 \PY{k}{for} \PY{n}{xi} \PY{o+ow}{in} \PY{n+nb}{range}\PY{p}{(}\PY{n}{xb}\PY{p}{,} \PY{n}{img}\PY{o}{.}\PY{n}{shape}\PY{p}{[}\PY{l+m+mi}{1}\PY{p}{]}\PY{o}{\PYZhy{}}\PY{n}{xb}\PY{p}{)}\PY{p}{:}
                     \PY{c+c1}{\PYZsh{}print(yi, xi)}
                     \PY{n}{data} \PY{o}{=} \PY{n}{img}\PY{p}{[}\PY{n}{yi}\PY{o}{\PYZhy{}}\PY{n}{yb}\PY{p}{:}\PY{n}{yi}\PY{o}{+}\PY{n}{yb}\PY{o}{+}\PY{l+m+mi}{1}\PY{p}{,} \PY{n}{xi}\PY{o}{\PYZhy{}}\PY{n}{xb}\PY{p}{:}\PY{n}{xi}\PY{o}{+}\PY{n}{xb}\PY{o}{+}\PY{l+m+mi}{1}\PY{p}{]}
                     \PY{n}{result}\PY{p}{[}\PY{n}{yi}\PY{p}{,} \PY{n}{xi}\PY{p}{]} \PY{o}{=} \PY{n}{func}\PY{p}{(}\PY{n}{data}\PY{p}{)}
             
             \PY{k}{if} \PY{n}{keep\PYZus{}size}\PY{p}{:}
                 \PY{k}{return} \PY{n}{result}
             \PY{k}{else}\PY{p}{:}
                 \PY{k}{return} \PY{n}{result}\PY{p}{[}\PY{n}{yb}\PY{p}{:}\PY{o}{\PYZhy{}}\PY{n}{yb}\PY{p}{,} \PY{n}{xb}\PY{p}{:}\PY{o}{\PYZhy{}}\PY{n}{xb}\PY{p}{]}
         
         \PY{c+c1}{\PYZsh{} Plot a set of images}
         \PY{k}{def} \PY{n+nf}{plot\PYZus{}images}\PY{p}{(}\PY{n}{image\PYZus{}array}\PY{p}{,} \PY{n}{cmap}\PY{o}{=}\PY{l+s+s1}{\PYZsq{}}\PY{l+s+s1}{gray}\PY{l+s+s1}{\PYZsq{}}\PY{p}{,} \PY{n}{title}\PY{o}{=}\PY{l+s+s1}{\PYZsq{}}\PY{l+s+s1}{\PYZsq{}}\PY{p}{)}\PY{p}{:}
             \PY{n}{f}\PY{p}{,} \PY{n}{ax} \PY{o}{=} \PY{n}{plt}\PY{o}{.}\PY{n}{subplots}\PY{p}{(}\PY{l+m+mi}{1}\PY{p}{,} \PY{n+nb}{len}\PY{p}{(}\PY{n}{image\PYZus{}array}\PY{p}{)}\PY{p}{,} \PY{n}{figsize}\PY{o}{=}\PY{p}{(}\PY{l+m+mi}{10}\PY{o}{*}\PY{n+nb}{len}\PY{p}{(}\PY{n}{image\PYZus{}array}\PY{p}{)}\PY{p}{,} \PY{l+m+mi}{10}\PY{p}{)}\PY{p}{)}
             \PY{n}{f}\PY{o}{.}\PY{n}{suptitle}\PY{p}{(}\PY{n}{title}\PY{p}{,} \PY{n}{fontsize}\PY{o}{=}\PY{l+m+mi}{50}\PY{p}{)}
             
             \PY{k}{if} \PY{n+nb}{len}\PY{p}{(}\PY{n}{image\PYZus{}array}\PY{p}{)} \PY{o}{==} \PY{l+m+mi}{1}\PY{p}{:}
                 \PY{n}{ax} \PY{o}{=} \PY{p}{[}\PY{n}{ax}\PY{p}{]}
                 
             \PY{k}{for} \PY{n}{a}\PY{p}{,} \PY{n}{img} \PY{o+ow}{in} \PY{n+nb}{zip}\PY{p}{(}\PY{n}{ax}\PY{p}{,} \PY{n}{image\PYZus{}array}\PY{p}{)}\PY{p}{:}
                 \PY{n}{a}\PY{o}{.}\PY{n}{matshow}\PY{p}{(}\PY{n}{img}\PY{p}{,} \PY{n}{cmap}\PY{o}{=}\PY{n}{cmap}\PY{p}{)}
                 \PY{n}{a}\PY{o}{.}\PY{n}{get\PYZus{}xaxis}\PY{p}{(}\PY{p}{)}\PY{o}{.}\PY{n}{set\PYZus{}visible}\PY{p}{(}\PY{k+kc}{False}\PY{p}{)}
                 \PY{n}{a}\PY{o}{.}\PY{n}{get\PYZus{}yaxis}\PY{p}{(}\PY{p}{)}\PY{o}{.}\PY{n}{set\PYZus{}visible}\PY{p}{(}\PY{k+kc}{False}\PY{p}{)}
                 
             \PY{k}{return} \PY{n}{f}
             
         \PY{c+c1}{\PYZsh{} Calculate the entropy of a image subregion}
         \PY{k}{def} \PY{n+nf}{global\PYZus{}entropie}\PY{p}{(}\PY{n}{data}\PY{p}{)}\PY{p}{:}
             \PY{k}{return} \PY{n}{np}\PY{o}{.}\PY{n}{sum}\PY{p}{(}\PY{n}{np}\PY{o}{.}\PY{n}{log}\PY{p}{(}\PY{n}{data}\PY{p}{)}\PY{o}{*}\PY{n}{data}\PY{o}{*}\PY{o}{\PYZhy{}}\PY{l+m+mi}{1}\PY{p}{)}
         
         \PY{c+c1}{\PYZsh{} Normalize an images}
         \PY{k}{def} \PY{n+nf}{normalize}\PY{p}{(}\PY{n}{img}\PY{p}{)}\PY{p}{:}
             \PY{k}{return} \PY{p}{(}\PY{n}{img} \PY{o}{\PYZhy{}} \PY{n}{img}\PY{o}{.}\PY{n}{min}\PY{p}{(}\PY{p}{)}\PY{p}{)}\PY{o}{/}\PY{n}{img}\PY{o}{.}\PY{n}{max}\PY{p}{(}\PY{p}{)}
         
         \PY{c+c1}{\PYZsh{} Calculate the probability image for given data}
         \PY{k}{def} \PY{n+nf}{image\PYZus{}to\PYZus{}pimage}\PY{p}{(}\PY{n}{data}\PY{p}{)}\PY{p}{:}
             \PY{n}{p}\PY{p}{,} \PY{n}{bins} \PY{o}{=} \PY{n}{np}\PY{o}{.}\PY{n}{histogram}\PY{p}{(}\PY{n}{data}\PY{p}{,} \PY{n+nb}{range}\PY{o}{=}\PY{p}{(}\PY{l+m+mf}{0.0}\PY{p}{,} \PY{l+m+mf}{1.0}\PY{p}{)}\PY{p}{,} \PY{n}{density}\PY{o}{=}\PY{k+kc}{True}\PY{p}{)}
             \PY{n}{p} \PY{o}{=} \PY{n}{p} \PY{o}{/} \PY{n+nb}{sum}\PY{p}{(}\PY{n}{p}\PY{p}{)}
             \PY{c+c1}{\PYZsh{} Calculate into which bin a pixel falls,}
             \PY{c+c1}{\PYZsh{} making sure the bin index has a corresponding probability}
             \PY{n}{pi} \PY{o}{=} \PY{n}{np}\PY{o}{.}\PY{n}{digitize}\PY{p}{(}\PY{n}{data}\PY{p}{,} \PY{n}{bins}\PY{p}{[}\PY{l+m+mi}{0}\PY{p}{:}\PY{o}{\PYZhy{}}\PY{l+m+mi}{1}\PY{p}{]}\PY{p}{)}\PY{o}{\PYZhy{}}\PY{l+m+mi}{1}
             \PY{k}{return} \PY{n}{p}\PY{p}{[}\PY{n}{pi}\PY{p}{]}
             
             
         \PY{c+c1}{\PYZsh{} First calculate the probability image for given data,}
         \PY{c+c1}{\PYZsh{}and than calculate its entropy}
         \PY{k}{def} \PY{n+nf}{local\PYZus{}entropy}\PY{p}{(}\PY{n}{data}\PY{p}{)}\PY{p}{:}
             \PY{n}{pdata} \PY{o}{=} \PY{n}{image\PYZus{}to\PYZus{}pimage}\PY{p}{(}\PY{n}{data}\PY{p}{)}
             \PY{k}{return} \PY{n}{global\PYZus{}entropie}\PY{p}{(}\PY{n}{pdata}\PY{p}{)}
         
         \PY{c+c1}{\PYZsh{} Calculate entropy based on the histogram of the dataset}
         \PY{k}{def} \PY{n+nf}{hist\PYZus{}entropy}\PY{p}{(}\PY{n}{data}\PY{p}{)}\PY{p}{:}
             \PY{n}{p}\PY{p}{,} \PY{n}{bins} \PY{o}{=} \PY{n}{np}\PY{o}{.}\PY{n}{histogram}\PY{p}{(}\PY{n}{data}\PY{p}{,} \PY{n}{density}\PY{o}{=}\PY{k+kc}{True}\PY{p}{)}
             \PY{n}{normp} \PY{o}{=} \PY{n}{p} \PY{o}{/} \PY{n}{p}\PY{o}{.}\PY{n}{sum}\PY{p}{(}\PY{p}{)}
             \PY{k}{return} \PY{n}{np}\PY{o}{.}\PY{n}{nansum}\PY{p}{(}\PY{n}{np}\PY{o}{.}\PY{n}{log}\PY{p}{(}\PY{n}{normp}\PY{p}{)}\PY{o}{*}\PY{n}{normp}\PY{o}{*}\PY{o}{\PYZhy{}}\PY{l+m+mi}{1}\PY{p}{)}
\end{Verbatim}


    \begin{Verbatim}[commandchars=\\\{\}]
{\color{incolor}In [{\color{incolor}6}]:} \PY{n}{f} \PY{o}{=} \PY{n}{plot\PYZus{}images}\PY{p}{(}\PY{p}{[}\PY{n}{gimage}\PY{p}{,}\PY{p}{]}\PY{p}{,} \PY{n}{title}\PY{o}{=}\PY{l+s+s1}{\PYZsq{}}\PY{l+s+s1}{Original grayscale image}\PY{l+s+s1}{\PYZsq{}}\PY{p}{)}
\end{Verbatim}


    \begin{center}
    \adjustimage{max size={0.9\linewidth}{0.9\paperheight}}{output_6_0.png}
    \end{center}
    { \hspace*{\fill} \\}
    
    \begin{Verbatim}[commandchars=\\\{\}]
{\color{incolor}In [{\color{incolor}11}]:} \PY{n}{pimage} \PY{o}{=} \PY{n}{image\PYZus{}to\PYZus{}pimage}\PY{p}{(}\PY{n}{gimage}\PY{p}{)}
         \PY{n}{global\PYZus{}entropy} \PY{o}{=} \PY{n}{convolute\PYZus{}image}\PY{p}{(}\PY{n}{pimage}\PY{p}{,} \PY{n}{global\PYZus{}entropie}\PY{p}{)}
         \PY{n}{f} \PY{o}{=} \PY{n}{plot\PYZus{}images}\PY{p}{(}\PY{n}{global\PYZus{}entropy}\PY{p}{,} \PY{n}{title}\PY{o}{=}\PY{l+s+s1}{\PYZsq{}}\PY{l+s+s1}{Global distribution based Entropy}\PY{l+s+s1}{\PYZsq{}}\PY{p}{)}
\end{Verbatim}


    \begin{center}
    \adjustimage{max size={0.9\linewidth}{0.9\paperheight}}{output_7_0.png}
    \end{center}
    { \hspace*{\fill} \\}
    
    \begin{Verbatim}[commandchars=\\\{\}]
{\color{incolor}In [{\color{incolor}12}]:} \PY{n}{subregion\PYZus{}entropy} \PY{o}{=} \PY{n}{convolute\PYZus{}image}\PY{p}{(}\PY{n}{gimage}\PY{p}{,} \PY{n}{local\PYZus{}entropy}\PY{p}{)}
         \PY{n}{f} \PY{o}{=} \PY{n}{plot\PYZus{}images}\PY{p}{(}\PY{n}{subregion\PYZus{}entropy}\PY{p}{,} \PY{n}{title}\PY{o}{=}\PY{l+s+s1}{\PYZsq{}}\PY{l+s+s1}{Sub region distribution based Entropy}\PY{l+s+s1}{\PYZsq{}}\PY{p}{)}
\end{Verbatim}


    \begin{center}
    \adjustimage{max size={0.9\linewidth}{0.9\paperheight}}{output_8_0.png}
    \end{center}
    { \hspace*{\fill} \\}
    
    \begin{Verbatim}[commandchars=\\\{\}]
{\color{incolor}In [{\color{incolor}9}]:} \PY{n}{himgs} \PY{o}{=} \PY{n}{convolute\PYZus{}image}\PY{p}{(}\PY{n}{gimage}\PY{p}{,} \PY{n}{hist\PYZus{}entropy}\PY{p}{)}
        \PY{n}{f} \PY{o}{=} \PY{n}{plot\PYZus{}images}\PY{p}{(}\PY{n}{himgs}\PY{p}{,} \PY{n}{title}\PY{o}{=}\PY{l+s+s1}{\PYZsq{}}\PY{l+s+s1}{Local histogram based Entropy}\PY{l+s+s1}{\PYZsq{}}\PY{p}{)}
\end{Verbatim}


    \begin{Verbatim}[commandchars=\\\{\}]
/Users/manuel.pasieka/anaconda3/envs/py3/lib/python3.6/site-packages/ipykernel\_launcher.py:70: RuntimeWarning: divide by zero encountered in log
/Users/manuel.pasieka/anaconda3/envs/py3/lib/python3.6/site-packages/ipykernel\_launcher.py:70: RuntimeWarning: invalid value encountered in multiply

    \end{Verbatim}

    \begin{center}
    \adjustimage{max size={0.9\linewidth}{0.9\paperheight}}{output_9_1.png}
    \end{center}
    { \hspace*{\fill} \\}
    
    \hypertarget{anuxe1lisis}{%
\section{Análisis}\label{anuxe1lisis}}

Tres distintas maneras de calcular la entropia de imagen fuen aplicados
* \textbf{Global distribution based entropy}: Calculaba la distribución
de los pixeles para el imagen entero, y la entropia en funcion de las
probabilidades de la region del kernel * \textbf{Sub region distribution
based entropy}: Calcula la distribución de pixeles y la Entropie para de
cada region del kernel independiente * \textbf{Histogram based entropy}:
Calcula un histograma para cada region del kernel y utiliza la
distribución de este histograma para calcular la Entropia

Desafortunamente, ninguna de las maneras de calcular la entropia, genera
regiones de entropia distina para los distintos `paterns'. Así la
analsis de entropia no ayuda en detectar las destintas texturas, o en
hacer una segmentación del imagen.


    % Add a bibliography block to the postdoc
    
    
    
    \end{document}
